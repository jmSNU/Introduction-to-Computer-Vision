
\documentclass[10pt]{article}
\usepackage[utf8]{inputenc}
\usepackage{kotex}
\usepackage{graphicx}
\usepackage{subfigure}
\usepackage{titling}
\setlength{\droptitle}{-2cm}
\usepackage{array}
\usepackage{amssymb}
\usepackage{amsmath}
\usepackage{siunitx} 
\usepackage{enumerate} 
\usepackage{pgfplots}
\usepackage{pgfplotstable}
\usepackage{tikz,pgfplots}
\usepackage{wasysym}
\usepackage{geometry}
\usepackage{authblk}
\usepackage{kotex}
\usepackage{bibunits}
\usepackage{tabularx}
\usepackage{hyperref}
\usepackage{pythonhighlight}

\geometry{
    a4paper,
    total={170mm,257mm},
    left=20mm,
    top=20mm,
}

\title{\textbf{Introduction to Computer Vision : HW 1}}
\author{Jeong Min Lee}

\begin{document}
\maketitle
\section*{Notation}
\begin{enumerate}
    \item $O(n)$ : Orthogonal group with dimension $n$.
    \item $r = \min (m,n)$ : From $\mathbf{A = U\Sigma V^T}$, $\mathbf{U} \in O(m)$, while $\mathbf{V} \in O(n)$
    \item $\Sigma$ : diagonal matrix whose entries are $\sigma_i$
\end{enumerate}
\section{Camera Calibration}
\subsection*{a}
From the result of SVD, 
\begin{equation}
    \mathbf{AV=U\Sigma}
\end{equation}
Since $\mathbf{V} \in O(n)$, the column vectors of $\mathbf{V}$, denoting $\mathbf{v_i}$ , can form the basis of $\mathbb{R}^n$. That is, $\forall \mathbf{p} \in \mathbb{R}^n, \mathbf{p} = \sum_i a_i \mathbf{v_i},\text{ for } a_i \in \mathbb{R}$.
This results in $\mathbf{Ap} =\sum_i a_i \mathbf{Av_i} = \sum_i a_i\sigma_i \mathbf{u_i}$. Thus,
\begin{equation}
    \lVert \mathbf{Av}\rVert^2 = \sum_i a_i^2\sigma_i^2
    \label{eqn2}
\end{equation}
It is a convention to make the diagonal entries of $\mathbf{\Sigma}$, $\sigma_i$, be ordered, that is, $\sigma_1 \ge \sigma_2 \ge \cdots \ge \sigma_r$.
Thus, to minimize equation \ref{eqn2}, $a_1 = a_2 = \cdots = a_{r-1} = 0$ and $a_r = 1$, since $\mathbf{p}$ is normalized. This implies $\mathbf{p = v_n}$.(note that $r = n$ since we assumed over-determined system.)

\subsection*{b} 
The following code is the implmentation to calculate the camera matrix for given corresponding points. 
I truncated the data load code in the following code for readability. Please refer to my jupyter note submission to get the whole implementation.
\begin{python}
x = ... #(2D points & truncated due to redundancy)

X = ... #(3D points & truncated due to redundancy)

A = []
for i in range(len(X)):
    xx = X[i]
    u,v = x[i]
    A.append(
        [xx[0], xx[1], xx[2],1,0,0,0,0,-u*xx[0], -u*xx[1], -u*xx[2], -u]
    )
    A.append(
        [0,0,0,0,xx[0], xx[1], xx[2],1,-v*xx[0], -v*xx[1], -v*xx[2], -v ]
    )
import numpy as np
A = np.array(A)
[U,S,V] = np.linalg.svd(A)
p = V[-1,:]
P_1 = p.reshape((3,4))
print(P_1)

\end{python}
The result of the code above is
\begin{python}
[[ 3.09963996e-03  1.46204548e-04 -4.48497465e-04 -9.78930678e-01]
 [ 3.07018252e-04  6.37193664e-04 -2.77356178e-03 -2.04144405e-01]
 [ 1.67933533e-06  2.74767684e-06 -6.83964827e-07 -1.32882928e-03]]
\end{python}

\subsection*{c}
Before determining $\mathbf{P}$, I proved that $\mathbf{p} = \mathbf{A}^\dagger \mathbf{b}$.
Consider SVD of $\mathbf{A = U\Sigma V^T} \in \mathbb{R}^{m\times n}$, where $\mathbf{U}\in O(m), \mathbf{\Sigma} \in \mathbb{R}^{m\times n}, \mathbf{V} \in O(n)$.
\begin{equation}
    \lVert \mathbf{Ap-b} \rVert^2 = \lVert \mathbf{U\Sigma V^Tp-b} \rVert^2 = \lVert \mathbf{\Sigma V^Tp - U^Tb} \rVert \quad (\because \mathbf{U}\in O(m))
\end{equation}
Let $\mathbf{V^Tp = q}, \mathbf{U^Tb = r}$ and rewrite the problem as follow.
\begin{equation}
    \text{Find } \arg\min_{\mathbf{q}} \lVert \mathbf{\Sigma q} - \mathbf{r}\rVert^2
\end{equation}
This problem can be solved as follow.
\begin{equation}
    \lVert \mathbf{\Sigma q - r}\rVert^2 = \sum_{i=1}^r (\sigma_iq_i - r_i)^2 \text{ where } r = \min \left\{m,n\right\}
\end{equation}
We can uniquely select the $q_i = r_i/\sigma_i$, for $i = 1,\cdots, r$, or $\mathbf{q = \Sigma^{-1}\mathbf{r}}$, to minimize this expression.
This implies $\mathbf{p = V\Sigma^{-1}U^Tb}$. Thus, proof is done, noting that $\mathbf{A^\dagger = (A^TA)^{-1}A^T = (V\Sigma^T\Sigma V^T)^{-1}\Sigma^TU^T = V\Sigma^{-1}U^T}$

% If $\mathbf{p}$ is the solution of $\mathbf{Ap = b}$, then $\mathbf{\Sigma q = r}$, that is $\sigma_i {q}_i = {r}_i$ for $i = 1,\cdots, r$(Note, $r = \min\left\{m,n\right\}$).
% Since the orthogonal transformation conserves the norm, $\mathbf{p}$ has a same norm to $\mathbf{q}$, $\lVert \mathbf{p}\rVert = \lVert \mathbf{q} \rVert$.
% Therefore, to minimize $\lVert \mathbf{p} \rVert$, we have to minimize $\lVert \mathbf{q}\rVert$. This problem can be formulated as follow:
% \begin{equation}
%     \arg \min_{\mathbf{q}} \lVert \mathbf{q} \rVert^2 \text{ where } \sigma_i q_i = r_i \text{ for } \forall i \in \left\{1,2,\cdots,r\right\}
% \end{equation}
% The solution of the problem above must satisfy $q_{r+1}, \cdots , q_n = 0$. Therefore, by denoting $\Sigma^{-1} \\ = \text{diag}(\sigma_1^{-1}, \cdots, \sigma_r^{-1},0,\cdots,0) \in \mathbb{R}^{m \times n}$,
% \begin{equation}
%     \mathbf{q = V^Tp = \Sigma^{-1}U^Tb}
% \end{equation}
% This implies 
% \begin{equation}
%     \mathbf{p = V\Sigma^{-1}U^Tb = A^\dagger b} 
% \end{equation}
% 
I used following code to calculate the camera matrix.
\begin{python}
def pseudo_inverse(A):
    return np.linalg.pinv(A)

A = []
b = []
for i in range(len(X)):
    xx = X[i]
    u,v = x[i]
    A.append(
        [xx[0], xx[1], xx[2],1,0,0,0,0,-u*xx[0], -u*xx[1], -u*xx[2]]
    )
    A.append(
        [0,0,0,0,xx[0], xx[1], xx[2],1,-v*xx[0], -v*xx[1], -v*xx[2]]
    )
    b.append(u)
    b.append(v)

A = np.array(A)
b = np.array(b)
p = np.matmul(pseudo_inverse(A),b)
p = np.append(p,1)
P_2 = p.reshape((3,4))
print(P_2)
\end{python}
The result of the code above is following.
\begin{python}
[[-2.33259098e+00 -1.09993113e-01  3.37413916e-01  7.36673920e+02]
 [-2.31050254e-01 -4.79506029e-01  2.08717636e+00  1.53627756e+02]
 [-1.26379606e-03 -2.06770917e-03  5.14635233e-04  1.00000000e+00]]
\end{python}

I used following code to check whether the result of problem 1-(b) and problem 1-(c) agree by setting $m_{34} = 1$ in the camera matrix calculated on problem 1-(b).
Since the result of following code is similar to camera matrix above, I can verify that both methodology are equivalent. 
\begin{python}
print(P_2[-1,-1]/P_1[-1,-1]*P_1)
\end{python}
The reason they are not exactly same is the floating point number error. 
\begin{python}
print(P_2[-1,-1]/P_1[-1,-1]*P_1 - P_2)
\end{python}
The code above returns the following matrix.
\begin{python}
[[-1.86427717e-05 -3.19670417e-05  9.93167477e-05  1.26469280e-02]
 [ 6.08843664e-06 -9.04052567e-06  4.56969897e-05 -4.92271847e-04]
 [ 2.54860563e-08 -3.33770907e-08  7.71079873e-08 -1.11022302e-16]]
\end{python}
As the (3,4) element of the matrix above is almost the accuracy limit of floating number system, there is some deviation when scaling of camera matrix in problem 1-(b).\footnote{The assertion that a 64-bit computer system is restricted to a maximum precision of 16 decimal digits is commonly based on the standard double-precision floating-point format, adhering to the IEEE 754 standard.}
This is common phenomenon in computer science.
\end{document}