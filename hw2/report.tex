
\documentclass[10pt]{article}
\usepackage[utf8]{inputenc}
\usepackage{kotex}
\usepackage{graphicx}
\usepackage{subfigure}
\usepackage{titling}
\setlength{\droptitle}{-2cm}
\usepackage{array}
\usepackage{amssymb}
\usepackage{amsmath}
\usepackage{siunitx} 
\usepackage{enumerate} 
\usepackage{pgfplots}
\usepackage{pgfplotstable}
\usepackage{tikz,pgfplots}
\usepackage{wasysym}
\usepackage{geometry}
\usepackage{authblk}
\usepackage{kotex}
\usepackage{bibunits}
\usepackage{tabularx}
\usepackage{hyperref}
\usepackage{pythonhighlight}

\geometry{
    a4paper,
    total={170mm,257mm},
    left=20mm,
    top=20mm,
}

\title{\textbf{Introduction to Computer Vision : HW 2}}
\author{Jeong Min Lee}

\begin{document}
\maketitle

\section*{1}
\subsection*{(a) Derive the 2D polar line representation: $\rho = x\cos\theta +y\sin\theta$}
Using the geometrical insight, one can easily understand the following relations.
\begin{align*}
    m &= \frac{1}{\tan\theta}\\
    b &= \frac{\rho}{\sin\theta}
\end{align*}
Inserting the relations above into $y = mx+b$,
\begin{align*}
    y &= -\frac{x\cos\theta}{\sin\theta} + \frac{\rho}{\sin\theta}\\
    \iff \rho &= x\cos\theta + y\sin\theta
\end{align*}
Thus, proof is done.
\subsection*{(b) Can you derive the polar representation of planes in 3D?}
It is not obvious what the term 'polar representation in 3D' means, since in 3D space, we don't have polar coordinates. 
Therefore, I regard the polar representation to the representation in spherical coordinates $(\rho,\theta, \phi)$, which corresponsds to the radius, elevation angle, and azimuthal angle, respectively.
\begin{figure}[!h]
    \begin{center}
        \includegraphics*[scale = 0.3]{./fig1.png}
    \end{center}
    \caption{The plane in 3D spherical coordinates.}
\end{figure}
Suppose the vector $OA$ is the normal vector of the arbitrary plane in 3D. Then, as figure 1 says, the vector $OA$ and $AB$ are orthogonal. 
Thus, noting that the vector $OA$ is, as a Cartesian coordinates, $(\rho \sin\theta\cos\varphi, \rho\sin\theta\sin\theta, \rho\cos\theta)$,
and the vector $AB$ is $(x - \rho \sin\theta\cos\varphi,y - \rho\sin\theta\sin\varphi,z - \rho\cos\theta)$, the orthogonal property gives the following relation.
\begin{equation}
    x\rho\sin\theta\cos\varphi - \rho^2\sin^2\theta\cos^2\varphi + y\rho\sin\theta\sin\varphi - \rho^2\sin^2\theta\sin^2\varphi + z\rho\cos\theta - \rho^2\cos^2\theta = 0
\end{equation}
Since $\rho^2\sin^2\theta\cos^2\varphi + \rho^2\sin^2\theta\sin^2\varphi + \rho^2\cos^\theta = \rho^2$, the equation 1 gives the following representation.
\begin{equation}
    x\sin\theta\cos\varphi + y\rho\sin\theta\sin\varphi + z\cos\theta= \rho
\end{equation}
\section*{2}
\subsection*{Write your own program of the Generalized HT(GHT) algorithm.}
The following code is my own GHT algorithm. Please note that I tried to implement every function from scratch. 
Also, I used OpenCV package to load an image, perform Canny edge detection, and visualize the images with bounding boxes. 
The following code performs GHT for given target image and template image and visualized the result of GHT.
The output of this algorithm is visualized in next section.

\begin{python}
import cv2
import numpy as np
import matplotlib.pyplot as plt
from collections import defaultdict
MIN_TEHTA = 0
MAX_THETA = 360
THETA_STEP = 45
MIN_THRESH = 150
MAX_THRESH = MIN_THRESH*3
scale_lst = [0.8,0.9,1,1.1]

def convolution(image, filter):
    image_height, image_width = image.shape
    filter_height, filter_width = filter.shape

    # padding to preserve the dimension of output
    padding_height = filter_height // 2
    padding_width = filter_width // 2
    padded_image = np.pad(image, ((padding_height, padding_height), (padding_width, padding_width)), mode='constant')
    output = np.zeros_like(image)

    # convolution
    for i in range(image_height):
        for j in range(image_width):
            output[i, j] = np.sum(padded_image[i:i+filter_height, j:j+filter_width] * filter)
    return output

def sobel_filter(image):
    # sobel filter
    filter = np.array([[-1, 0, 1], [-2, 0, 2], [-1, 0, 1]])
    # calculate horizontal and vertical gradient of image
    new_image_x = convolution(image, filter)
    new_image_y = convolution(image, np.flip(filter.T, axis=0))

    return new_image_x, new_image_y

def gradient_orientation(image):
    # calculate the gradient given horizontal and vertical images
    dx, dy = sobel_filter(image)
    orientation = (np.arctan2(dy, dx) * 180 / np.pi) % 360  # normalize to [0, 360) degrees

    return orientation

def build_r_table(image, center):
    edges = cv2.Canny(image, MIN_THRESH, MAX_THRESH)
    gradient = gradient_orientation(edges)  # orientation
    r_table = defaultdict(list)

    # build a R-table
    for (i, j), value in np.ndenumerate(edges):
        if value:
            # Calculate the rotation and scale invariant distance and angle
            r = np.sqrt((center[0] - i) ** 2 + (center[1] - j) ** 2)
            phi = (np.arctan2(center[0] - i, center[1] - j) - np.radians(gradient[i, j])) % (2 * np.pi)  # normalize to [0, 2*pi) radians
            r_table[gradient[i, j]].append((r, phi))

    return r_table

def non_max_suppression(boxes, overlapThresh):
    if len(boxes) == 0:
        return []
    if boxes.dtype.kind == "i":
        boxes = boxes.astype("float")
    pick = []
    x1 = boxes[:,0]
    y1 = boxes[:,1]
    x2 = boxes[:,2]
    y2 = boxes[:,3]
    area = (x2 - x1 + 1) * (y2 - y1 + 1)
    idxs = np.argsort(y2)
    while len(idxs) > 0:
        last = len(idxs) - 1
        i = idxs[last]
        pick.append(i)
        xx1 = np.maximum(x1[i], x1[idxs[:last]])
        yy1 = np.maximum(y1[i], y1[idxs[:last]])
        xx2 = np.minimum(x2[i], x2[idxs[:last]])
        yy2 = np.minimum(y2[i], y2[idxs[:last]])
        w = np.maximum(0, xx2 - xx1 + 1)
        h = np.maximum(0, yy2 - yy1 + 1)
        overlap = (w * h) / area[idxs[:last]]
        idxs = np.delete(idxs, np.concatenate(([last], np.where(overlap > overlapThresh)[0])))
    return boxes[pick].astype("int")

def accumulate_gradients(r_table, grayImage):
    edges = cv2.Canny(grayImage,0,360)
    gradient = gradient_orientation(edges)
    
    accumulator = np.zeros(grayImage.shape)
    for (i,j),value in np.ndenumerate(edges):
        if value:
            for r in r_table[gradient[i,j]]:

                # retrieve the information of reference point from R-table
                accum_i, accum_j = i+r[0], j+r[1]
                accum_i, accum_j = int(accum_i), int(accum_j)

                # majority votes
                if accum_i < accumulator.shape[0] and accum_j < accumulator.shape[1]:
                    accumulator[accum_i, accum_j] += 1
                    
    return accumulator

def main():
    template_image_path = "../hw2/template_image.png"
    target_image_path = "../hw2/target_image.png"

    # Read template and target images
    template_image = cv2.imread(template_image_path, cv2.IMREAD_GRAYSCALE)
    target_image = cv2.imread(target_image_path, cv2.IMREAD_GRAYSCALE)

    # Perform Canny edge detection on template and target images
    template_edges = cv2.Canny(template_image, 50, 150)
    target_edges = cv2.Canny(target_image, 50, 150)

    center = (template_edges.shape[0] // 2, template_edges.shape[1] // 2)
    r_table = build_r_table(template_edges, center)
    accumulator = accumulate_gradients(r_table, target_edges)

    plt.figure(figsize=(10, 10))

    # Plot reference image (template)
    plt.subplot(221)
    plt.imshow(template_image, cmap='gray')
    plt.plot(center[1], center[0], 'ro')
    plt.title('Reference image')
    plt.axis('off')

    # Plot query image (target)
    plt.subplot(222)
    plt.imshow(target_image, cmap='gray')
    plt.title('Query image')
    plt.axis('off')

    # Plot accumulator
    plt.subplot(223)
    plt.imshow(accumulator, cmap='jet')
    plt.title('Accumulator')
    plt.axis('off')

    plt.subplot(224)
    threshold = 0.8 * np.max(accumulator)  # Adjust threshold as needed
    peaks = np.argwhere(accumulator > threshold)
    img_color = cv2.cvtColor(target_image, cv2.COLOR_GRAY2BGR)  # Convert image to color
    all_boxes = []
    for peak in peaks:
        plt.plot(peak[1], peak[0], 'r.')  # Plot peak in reverse order due to numpy array indexing
        tmp_center = (int(peak[1]), int(peak[0]))
        bbox_width = int(template_edges.shape[1])
        bbox_height = int(template_edges.shape[0])
        bbox_x = max(tmp_center[0] - bbox_width // 2, 0)
        bbox_y = max(tmp_center[1] - bbox_height // 2, 0)

        boxes = [bbox_x, bbox_y, bbox_x + bbox_width, bbox_y + bbox_height]
        all_boxes.append(boxes)
        
    nms_boxes = non_max_suppression(np.array(all_boxes),0.5)
    for box in nms_boxes:
        cv2.rectangle(img_color, (box[0],box[1]),(box[2],box[3]),(255,0,0),2)
        
    # Plot detection results (overlay edge template)
    plt.imshow(img_color)
    plt.title('Detection')
    plt.axis('off')
    plt.tight_layout()
    plt.show()
    
if __name__ == "__main__":
    main()
\end{python}

\subsection*{Test your algorithm on the given template image and the target image. You need to convert both images into edge images.}
\subsection*{Show your detection results and analyze them.}
\begin{figure}[!h]
    \begin{center}
        \includegraphics*[scale = 0.4]{../hw2/fig2.png}
    \end{center}
    \caption{The output of GHT algorithm described the section above. The red dot in first image denotes the predefined reference point of template, while that in last image indicate the maximum of accumulator. }
\end{figure}
As figure 2 shows, the red dot which indicates the point that matches to the template is located on the proper position where the target image patch agrees to the reference template. 
As the red bounding box illlustrated, this implies that the implementation of GHT above properly reflects the main intention of original paper of GHT.
However, the code above cannot catch the resized and rotated template in target image since the parameters that indicate those varitaion is not included in the code above.
To solve this problem, I introduced the rotation and scaling paramters on my algorithm. This will be discussed in the next section.  

\subsection*{Can you make your GHT algorithm scale and rotation invariant? How? Show the results.}

\begin{python}

def accumulate_gradients(r_table, grayImage):
    edges = cv2.Canny(grayImage, MIN_THRESH, MAX_THRESH)
    gradient = gradient_orientation(edges)
    accumulator = np.zeros((grayImage.shape[0], grayImage.shape[1], len(scale_lst), len(range(MIN_TEHTA, MAX_THETA, THETA_STEP))), dtype=np.int64)

    for (i, j), value in np.ndenumerate(edges):
        if value:
            for r, phi in r_table[gradient[i, j]]:
                i_p = r*np.cos(phi)
                j_p = r*np.sin(phi)
                for theta in range(MIN_TEHTA, MAX_THETA, THETA_STEP):
                    theta_rad = np.radians(theta)
                    for scale in scale_lst:
                        i_c = int(i - scale*(i_p*np.cos(theta_rad) - j_p*np.sin(theta_rad)))  # calculate the new center (i_c, j_c)
                        j_c = int(j - scale*(i_p*np.sin(theta_rad) + j_p*np.cos(theta_rad)))
                        if 0 <= i_c < grayImage.shape[0] and 0 <= j_c < grayImage.shape[1]:  # check if the new center is within the image
                            accumulator[i_c, j_c, scale_lst.index(scale), theta // THETA_STEP] += 1

    return accumulator

def main():
    template_image_path = "../hw2/template_image.png"
    target_image_path = "../hw2/target_image.png"

    # Read template and target images
    template_image = cv2.imread(template_image_path, cv2.IMREAD_GRAYSCALE)
    target_image = cv2.imread(target_image_path, cv2.IMREAD_GRAYSCALE)

    # Perform Canny edge detection on template and target images
    template_edges = cv2.Canny(template_image,  MIN_THRESH, MAX_THRESH)
    target_edges = cv2.Canny(target_image,  MIN_THRESH, MAX_THRESH)

    center = (template_edges.shape[0] // 2, template_edges.shape[1] // 2)
    r_table = build_r_table(template_edges, center)
    accumulator = accumulate_gradients(r_table, target_image)

    plt.figure(figsize=(10, 10))
    plt.subplot(121)
    plt.imshow(accumulator[:,:,0,0], cmap='jet')
    plt.title('Accumulator')
    plt.axis('off')
    # Plot detection results (overlay edge template)
    plt.subplot(122)
    threshold = 0.7 * np.max(accumulator)  # Adjust threshold as needed
    peaks = np.argwhere(accumulator > threshold)

    img_color = cv2.cvtColor(target_image, cv2.COLOR_GRAY2BGR)  # Convert image to color
    all_boxes = []

    for peak in peaks:
        plt.plot(peak[1], peak[0], 'r.')  # Plot peak in reverse order due to numpy array indexing
        scale, rotation = scale_lst[peak[2]], MIN_TEHTA + THETA_STEP*peak[3]

        tmp_center = (int(peak[1]), int(peak[0]))
        bbox_width = int(template_edges.shape[1] * scale)
        bbox_height = int(template_edges.shape[0] * scale)
        bbox_x = max(tmp_center[0] - bbox_width // 2, 0)
        bbox_y = max(tmp_center[1] - bbox_height // 2, 0)

        boxes = [bbox_x, bbox_y, bbox_x + bbox_width, bbox_y + bbox_height]
        all_boxes.append(boxes)
        
    nms_boxes = non_max_suppression(np.array(all_boxes),0.5)

    for box in nms_boxes:
        cv2.rectangle(img_color, (box[0],box[1]), (box[2],box[3]), (255,0,0),2)
        
    plt.imshow(img_color)
    plt.title('Detection')
    plt.axis('off')
    plt.tight_layout()
    plt.show()

if __name__ == "__main__":
    main()
\end{python}
\begin{figure}[!h]
    \begin{center}
        \includegraphics*[scale = 0.5]{../hw2/fig3.png}
    \end{center}
    \caption{The output of scale-rotation invariant GHT algorithm described the section above. The red dot in first image denotes the predefined reference point of template, while that in last image indicate the maximum of accumulator. }
\end{figure}
As illustrated in Figure 3, each airplane within the target image has been successfully detected. The scale-rotation invariant Generalized Hough Transform (GHT), being more sensitive to noise, results in multiple bounding boxes indicating the same object. I hypothesize that this issue could be mitigated through the preprocessing of the target and template images. In this particular project, I did not apply any preprocessing measures to the images. However, it's worth noting that airplane shadows have a tendency to be misidentified as the airplane itself. 
In an attempt to rectify this, I optimized a number of parameters, including the minimum and maximum thresholds of Canny(), the threshold of the accumulator map, and the threshold in non-max suppression. Despite these adjustments, as evidenced in Figure 3, there are still bounding boxes that mistakenly identify shadows as airplanes. I believe that the resolution of this issue lies in image processing rather than the GHT.
An important observation is that the bounding box has a greater propensity to erroneously detect a shadow as an airplane when the shadow is roughly the same size as the airplane. 
\end{document}